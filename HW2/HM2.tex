\documentclass[12pt,a4paper]{article}

\usepackage{amsthm}
\usepackage{amsmath}
\usepackage{amsfonts}
\usepackage{tikz}
\usepackage{wrapfig}
\usepackage{multicol}
\usepackage{color}

\usepackage[utf8]{inputenc}
\usepackage[english,russian]{babel}

\usetikzlibrary{automata, positioning, arrows}
%-------------------------------------------

\setlength{\textwidth}{7.0in}
\setlength{\oddsidemargin}{-0.35in}
\setlength{\topmargin}{-0.5in}
\setlength{\textheight}{9.0in}
\setlength{\parindent}{0.3in}

\newtheorem{task}{Задача}

\begin{document}

\tikzset{
  ->,
  >=stealth',
  node distance=3cm,
  every state/.style={thick, fill=gray!10},
  initial text=$ $,
}

\begin{flushright}
\textbf{
  ЛЭТИ, гр. 8303, Гришин Константин \\
  \today
}
\end{flushright}

\begin{center}
\textbf{Формальные языки} \\
\textbf{HW02}             \\
\end{center}

\noindent\textit{Решение 1.1}
$\{\omega \in \{0, 1, \dots, 9\}^* \mid \omega \text{ делится нацело на } 5 \}$ \\
\begin{figure}[ht]
  \centering
  \begin{tikzpicture}
    \node[state, initial, accepting](g){g};
    \node[state, right of=g](b){b};
    \draw (g) edge[loop above]       node{0, 5}     (g)
          (g) edge[bend left, above] node{not 0, 5} (b)
          (b) edge[loop right]       node{not 0, 5} (b)
          (b) edge[bend left, below] node{0, 5}     (g);
  \end{tikzpicture}
\end{figure} \\
Тривиальный случай, минимизация невозможна.
\qed
\vspace{1em}

\noindent\textit{Решение 1.2}
Язык положительных чисел с плавающей точкой.
Целая часть может отсутствовать, дробная часть может отсутствовать,
но не одновременно. Перед точкой может не быть ничего.
Лидирующих нулей быть не должно
\begin{itemize}
  \item $123.45, .45, 123$ -- числа
  \item $\varepsilon, ., abc, 123., 1.2.3.4, 007$ -- не числа
\end{itemize}
\begin{figure}[ht]
  \centering
  \begin{tikzpicture}
    \node[state] (start)                  [initial]   {s};
    \node[state] (dot)   [right of=start]             {$\cdot$};
    \node[state] (float) [right of=dot]   [accepting] {f};
    \node[state] (int)   [above of=dot]   [accepting] {i};
    \node[state] (sink)  [below of=dot]               {$\perp$};
    \draw (start) edge[above]      node               {\{$\cdot$\}}         (dot)
          (start) edge[above]      node[xshift=-1.5em]{\{1,...,9\}}         (int)
          (start) edge[below]      node[xshift=-1em]  {\{0\}}               (sink)
          (int)   edge[loop left]  node               {\{0,...,9\}}         (int)
          (int)   edge[left]       node               {\{$\cdot$\}}         (dot)
          (dot)   edge[left]       node               {\{$\cdot$\}}         (sink)
          (dot)   edge[above]      node               {\{0,...,9\}}         (float)
          (float) edge[loop right] node               {\{0,...,9\}}         (float)
          (float) edge[right]      node[xshift=1em]   {\{$\cdot$\}}         (sink)
          (sink)  edge[loop left]  node               {\{0,...,9,$\cdot$\}} (sink);
  \end{tikzpicture}
\end{figure}
Рассмотрим следующие пары неэквивалентых состояний:
\[\{
  \langle s,     i \rangle,
  \langle \cdot, i \rangle,
  \langle \perp, i \rangle,
  \langle s,     f \rangle,
  \langle \cdot, f \rangle,
  \langle \perp, f \rangle
\}\]
Пары
$\{
  \langle \perp, i \rangle,
  \langle \perp, f \rangle
\}$
позволяют найти новые неэквивалентные состояния:
\[\{
  \langle \perp, s     \rangle,
  \langle \perp, \cdot \rangle,
  \langle \cdot, s     \rangle,
  \langle i,     f     \rangle
\}\]
Рассмотренны все возможные классы эквивалентности,
экивалентных вершин не оказалось. Следовательно, автомат является минимальным.
\qed

\clearpage
\noindent\textit{Решение 1.3}
$\{\omega \in \{a, b\}^* \mid |\omega|_a \leq 3, |\omega|_b > 2\}$ \\
\begin{figure}[ht]
  \centering
  \begin{tikzpicture}
    \node[state] (start)                  [initial]   {S};
    \node[state] (01)    [right of=start]             {01};
    \node[state] (02)    [right of=01]                {02};
    \node[state] (0a)    [right of=02]    [accepting] {0a};

    \node[state] (10)    [below of=start]             {10};
    \node[state] (11)    [below of=01]                {11};
    \node[state] (12)    [below of=02]                {12};
    \node[state] (1a)    [below of=0a]    [accepting] {1a};

    \node[state] (20)    [below of=10]                {20};
    \node[state] (21)    [below of=11]                {21};
    \node[state] (22)    [below of=12]                {22};
    \node[state] (2a)    [below of=1a]    [accepting] {2a};

    \node[state] (30)    [below of=20]                {30};
    \node[state] (31)    [below of=21]                {31};
    \node[state] (32)    [below of=22]                {32};
    \node[state] (3a)    [below of=2a]    [accepting] {3a};

    \node[state] (sink)  at (4.5, -12)                {X};

    \draw (start) edge[left]       node {a} (10)
          (start) edge[above]      node {b} (01)
          
          (01)    edge[left]       node {a} (11)
          (01)    edge[above]      node {b} (02)

          (02)    edge[left]       node {a} (12)
          (02)    edge[above]      node {b} (0a)

          (0a)    edge[left]       node {a} (1a)
          (0a)    edge[loop right] node {b} (0a)

          (10)    edge[left]       node {a} (20)
          (10)    edge[above]      node {b} (11)
          
          (11)    edge[left]       node {a} (21)
          (11)    edge[above]      node {b} (12)

          (12)    edge[left]       node {a} (22)
          (12)    edge[above]      node {b} (1a)

          (1a)    edge[left]       node {a} (2a)
          (1a)    edge[loop right] node {b} (1a)

          (20)    edge[left]       node {a} (30)
          (20)    edge[above]      node {b} (21)
          
          (21)    edge[left]       node {a} (31)
          (21)    edge[above]      node {b} (22)

          (22)    edge[left]       node {a} (32)
          (22)    edge[above]      node {b} (2a)

          (2a)    edge[left]       node {a} (3a)
          (2a)    edge[loop right] node {b} (2a)

          (30)    edge[below]       node {a} (sink)
          (30)    edge[above]      node {b} (31)
          
          (31)    edge[left]       node {a} (sink)
          (31)    edge[above]      node {b} (32)

          (32)    edge[right]      node {a} (sink)
          (32)    edge[above]      node {b} (3a)

          (3a)    edge[below]       node {a} (sink)
          (3a)    edge[loop right] node {b} (3a)
          
          (sink)  edge[loop below] node {a, b} (sink);    
  \end{tikzpicture}
\end{figure} \\
Автомат состоит из $17$ вершин. Возможных пар $(17^2 - 17) / 2 = 136$.
Конечных состояний $4$, пар вершин, которые различимы $\varepsilon$,
$(17-4) * 4 = 52$. Следовательно для оставшихся $(136 - 52) = 84$ пар
должны быть нетривиальные последовательности символов, которые их различают:

\setlength{\columnseprule}{1pt}
\def\columnseprulecolor{\color{black}}

\begin{multicols}{5}
\noindent\small
(22, 11): "b" \\
(22, 20): "b" \\
(22, 12): "aab" \\
(22, 31): "b" \\
(22, 32): "ab" \\
(22, 30): "b" \\
(22, X):  "b" \\
(22, S):  "b" \\
(22, 01): "b" \\
(22, 10): "b" \\
(22, 21): "b" \\
(22, 02): "aab" \\
(20, 11): "bb" \\
(12, 11): "b" \\
(31, 11): "abb" \\
(11, 32): "b" \\
(11, 30): "bb" \\
(11, X):  "bb" \\
(11, S):  "bb" \\
(11, 01): "aaabb" \\
(11, 10): "bb" \\
(11, 21): "aabb" \\
(11, 02): "b" \\
(20, 12): "b" \\
(20, 31): "bb" \\
(20, 32): "b" \\
(20, 30): "abbb" \\
(20, X):  "bbb" \\
(20, S):  "aabbb" \\
(20, 01): "bb" \\
(20, 10): "aabbb" \\
(20, 21): "bb" \\
(20, 02): "b" \\
(31, 12): "b" \\
(12, 32): "ab" \\
(12, 30): "b" \\
(12, X):  "b" \\
(12, S):  "b" \\
(12, 01): "b" \\
(12, 10): "b" \\
(12, 21): "b" \\
(12, 02): "aaab" \\
(31, 32): "b" \\
(31, 30): "bb" \\
(31, X):  "bb" \\
(31, S):  "bb" \\
(31, 01): "abb" \\
(31, 10): "bb" \\
(31, 21): "abb" \\
(31, 02): "b" \\
(32, 30): "b" \\
(32, X):  "b" \\
(S, 32):  "b" \\
(32, 01): "b" \\
(32, 10): "b" \\
(21, 32): "b" \\
(32, 02): "ab" \\
(X, 30):  "bbb" \\
(S, 30):  "abbb" \\
(01, 30): "bb" \\
(30, 10): "abbb" \\
(21, 30): "bb" \\
(30, 02): "b" \\
(S, X):   "bbb" \\
(X, 01):  "bb" \\
(X, 10):  "bbb" \\
(21, X):  "bb" \\
(X, 02):  "b" \\
(1a, 2a): "aa" \\
(1a, 3a): "a" \\
(1a, 0a): "aaa" \\
(3a, 2a): "a" \\
(0a, 2a): "aa" \\
(S, 01):  "bb" \\
(S, 10):  "aaabbb" \\
(21, S):  "bb" \\
(S, 02):  "b" \\
(01, 10): "bb" \\
(21, 01): "aabb" \\
(01, 02): "b" \\
(21, 10): "bb" \\
(02, 10): "b" \\
(3a, 0a): "a" \\
(21, 02): "b"
\end{multicols}
\qed
\end{document}